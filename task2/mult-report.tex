\documentclass[12pt,a4paper,oneside,final]{article}
\usepackage{pgfplots}
% \pgfplotsset{width=7cm, compat=1.18}
\pgfplotsset{compat=1.9}
\usepackage[utf8]{inputenc}
\usepackage[russian]{babel}
\usepackage{graphicx} % \includegraphics
\usepackage{indentfirst}

\oddsidemargin = 0cm
\topmargin = -1.5cm
\textwidth = 16cm
\textheight = 24cm
\parindent = 0.5cm

\newcommand\Section[1]{
  \refstepcounter{section}
  \section*{\raggedright
    \arabic{section}. #1}
  \addcontentsline{toc}{section}{%
    \arabic{section}. #1}
}

\newcommand\Subsection[1]{
  \refstepcounter{subsection}
  \subsection*{\raggedright
    \arabic{section}.\arabic{subsection}. #1
  }
  \addcontentsline{toc}{subsection}{%
    \arabic{section}.\arabic{subsection}. #1}
}

\newcommand\Subsubsection[1]{
  \refstepcounter{subsubsection}
  \subsubsection*{\raggedright
    \arabic{section}.\arabic{subsection}.\arabic{subsubsection}. #1
  }
  \addcontentsline{toc}{subsubsection}{%
    \arabic{section}.\arabic{subsection}.\arabic{subsubsection}. #1}
}

\sloppy

\title{Задание 2 \\
  Анализ влияния кэша на операцию матричного
умножения \\
  Отчёт}
\author{Чижевский\,А.А.}
\date{2023}

\begin{document}

\maketitle

\Section{Постановка задачи}


Реализовать последовательный алгоритм матричного умножения и оценить влияние кэша на время выполнения программы.

\Section{Формат коммандной строки}

\begin{verbatim}
<binary> <матрица_а> <матрица_b> <матрица_c> <режим>
\end{verbatim}

\noindent
Режимы: 0 – ijk, 1 – ikj, 2 – kij, 3 – jik, 4 – jki, 5 – kji.

\Section{Спецификация системы}

\noindent
Процессор: Apple M1

\noindent
Число вычислительных ядер: 8

\Section{Результаты выполнения}

Размеры матриц: $300\times300, 500\times500, 1000\times1000$

Ниже приведены графики зависимости времени работы основного цикла програм-
мы (в секундах) от режима итерирования.

\begin{center}
    \begin{tikzpicture}
    \begin{axis}[
        scaled ticks = false,
        yticklabel style={
            /pgf/number format/precision=3,
            /pgf/number format/fixed,
            /pgf/number format/fixed zerofill,
          },
        legend pos = north west,
        xtick distance = 1,
    	xlabel = {Режим},
    	ylabel = {Время, с},
    	minor tick num = 0,
        xmin = 0,
        xmax = 5,
        ymin = 0.02,
        ymax = 0.07,
        width = 400,
        height = 320
    ]
    \legend{ 
	$msize\ 300\times300$
    };
    \addplot coordinates {
	    (0,0.033812) (1,0.021545) (2,0.021397) (3,0.038256) (4,0.065307) (5,0.066271)
    };
    
    \end{axis}
    \end{tikzpicture}
\end{center}

\begin{center}
    \begin{tikzpicture}
    \begin{axis}[
        scaled ticks = false,
        yticklabel style={
            /pgf/number format/precision=2,
            /pgf/number format/fixed,
            /pgf/number format/fixed zerofill,
          },
        legend pos = north west,
        xtick distance = 1,
    	xlabel = {Режим},
    	ylabel = {Время, с},
    	minor tick num = 0,
        xmin = 0,
        xmax = 5,
        ymin = 0.05,
        ymax = 0.95,
        width = 400,
        height = 320
    ]
    \legend{ 
	$msize\ 500\times500$
    };
    \addplot coordinates {
	    (0,0.112552) (1,0.067886) (2,0.06811) (3,0.115821) (4,0.882386) (5,0.843262)
    };
    
    \end{axis}
    \end{tikzpicture}
\end{center}

\begin{center}
    \begin{tikzpicture}
    \begin{axis}[
        scaled ticks = false,
        yticklabel style={
            /pgf/number format/precision=1,
            /pgf/number format/fixed,
            /pgf/number format/fixed zerofill,
          },
        legend pos = north west,
        xtick distance = 1,
    	xlabel = {Режим},
    	ylabel = {Время, с},
    	minor tick num = 0,
        xmin = 0,
        xmax = 5,
        ymin = 0,
        ymax = 5.5,
        width = 400,
        height = 320
    ]
    \legend{ 
	$msize\ 1000\times1000$
    };
    \addplot coordinates {
	    (0,0.992807) (1,0.381163) (2,0.408762) (3,0.98364) (4,5.14642) (5,5.15529)
    };
    
    \end{axis}
    \end{tikzpicture}
\end{center}

\end{document}
