\documentclass[12pt,a4paper,oneside,final]{article}
\usepackage[utf8]{inputenc}
\usepackage[russian]{babel}
\usepackage{graphicx} % \includegraphics
\usepackage{indentfirst}

\oddsidemargin = 0cm
\topmargin = -1.5cm
\textwidth = 16cm
\textheight = 24cm
\parindent = 0.5cm

\newcommand\Section[1]{
  \refstepcounter{section}
  \section*{\raggedright
    \arabic{section}. #1}
  \addcontentsline{toc}{section}{%
    \arabic{section}. #1}
}

\newcommand\Subsection[1]{
  \refstepcounter{subsection}
  \subsection*{\raggedright
    \arabic{section}.\arabic{subsection}. #1
  }
  \addcontentsline{toc}{subsection}{%
    \arabic{section}.\arabic{subsection}. #1}
}

\newcommand\Subsubsection[1]{
  \refstepcounter{subsubsection}
  \subsubsection*{\raggedright
    \arabic{section}.\arabic{subsection}.\arabic{subsubsection}. #1
  }
  \addcontentsline{toc}{subsubsection}{%
    \arabic{section}.\arabic{subsection}.\arabic{subsubsection}. #1}
}

\sloppy

\title{Задание 3 \\
  Подсчёт количества промахов в кэш для операции матричного умножения в зависимости от порядка итерирования \\
  Отчёт}
\author{Чижевский\,А.А.}
\date{2023}

\begin{document}

\maketitle

\Section{Постановка задачи}


При помощи PAPI снять значения аппаратных счетчиков промахов L1/L2 кэшей при выполнении операции умножения квадратных матриц. Сравнить полученные значения с теоретическими для каждого порядка итерирования.

\Section{Формат коммандной строки}

\begin{verbatim}
<binary> <матрица_а> <матрица_b> <матрица_c> <режим>
\end{verbatim}

\noindent
Режимы: 0 – ijk, 1 – ikj, 2 – kij, 3 – jik, 4 – jki, 5 – kji.

\Section{Спецификация системы}

\noindent
Процессор: POWER8NVL 4023MHz 

\noindent
Размеры кэшей 

\noindent
L1d cache:             64K

\noindent
L1i cache:             32K

\noindent
L2 cache:              512K

\noindent
Размер строки кэша L1: 128 байт

\Section{Результаты выполнения}

Для исследования размер матрицы был выбран $1000\times1000$.

\renewcommand{\arraystretch}{2}
\begin{center}
 \resizebox{16cm}{!} {
  \begin{tabular}[t]{|l|l|l|l|l|}
    \hline
      & L1 cache misses & L2 cache misses & L1 Theor & $\frac{L1Theor}{L1misses}$ \\
    \hline
    ijk & $2036251202$ & $31483206$ & $1031250000$ & $0.5064$ \\
    \hline
    ikj & $29454181$ & $115388$ & $62500000$ & $2.1219$\\
    \hline
    kij & $23760392$ & $1188211$ & $62500000$ & $2.6304$\\
    \hline
    jik & $2044664391$ & $31493958$ & $1031250000$ & $0.5044$\\
    \hline
    jki & $4295738183$ & $35928740$ & $2000000000$ & $0.4656$\\
    \hline
    kji & $4041456024$ & $36615068$ & $2000000000$ & $0.4949$\\
    \hline
  \end{tabular}
  }
\end{center}

\end{document}
