\documentclass[12pt,a4paper,oneside,final]{article}
\usepackage{pgfplots}
% \pgfplotsset{width=7cm, compat=1.18}
\pgfplotsset{compat=1.9}
\usepackage[utf8]{inputenc}
\usepackage[russian]{babel}
\usepackage{graphicx} % \includegraphics
\usepackage{indentfirst}

\oddsidemargin = 0cm
\topmargin = -1.5cm
\textwidth = 16cm
\textheight = 24cm
\parindent = 0.5cm

\newcommand\Section[1]{
  \refstepcounter{section}
  \section*{\raggedright
    \arabic{section}. #1}
  \addcontentsline{toc}{section}{%
    \arabic{section}. #1}
}

\newcommand\Subsection[1]{
  \refstepcounter{subsection}
  \subsection*{\raggedright
    \arabic{section}.\arabic{subsection}. #1
  }
  \addcontentsline{toc}{subsection}{%
    \arabic{section}.\arabic{subsection}. #1}
}

\newcommand\Subsubsection[1]{
  \refstepcounter{subsubsection}
  \subsubsection*{\raggedright
    \arabic{section}.\arabic{subsection}.\arabic{subsubsection}. #1
  }
  \addcontentsline{toc}{subsubsection}{%
    \arabic{section}.\arabic{subsection}.\arabic{subsubsection}. #1}
}

\sloppy

\title{Задание 5 \\
  Параллельный алгоритм DNS матричного умножения\\
  Отчёт}
\author{Чижевский\,А.А.}
\date{2023}

\begin{document}

\maketitle

\Section{Постановка задачи}

Реализовать параллельный 3D алгоритм DNA матричного умножения. Протестировать полученную программу на Polus. Составить графики для матриц размерности $1536\times1536, 3072\times3072$ и для числа процессов $n = \{1, 2^3 = 8, 3^3=27\}$.

\Section{Формат коммандной строки}

\begin{verbatim}
   mpisubmit.pl -p <Np> ./prog -- <матрица_а> <матрица_b> <матрица_c>
\end{verbatim}

\Section{Спецификация системы}

\noindent

\noindent
Процессор: POWER8NVL 4023MHz 

\noindent
Размеры кэшей 

\noindent
L1d cache:             64K

\noindent
L1i cache:             32K

\noindent
L2 cache:              512K

\noindent
Размер строки кэша L1: 128 байт

\noindent
Число вычислительных ядер: 8

\Section{Результаты выполнения}


Ниже приведены графики зависимости времени работы основного цикла программы (в секундах) от количества MPI процессов.

\begin{center}
    \begin{tikzpicture}
    \begin{axis}[
        scaled ticks = false,
        	xtick = {1, 8, 27},
        yticklabel style={
            /pgf/number format/precision=3,
            /pgf/number format/fixed,
            /pgf/number format/fixed zerofill,
          },
        legend pos = north east,
    	xlabel = {Количество процессов},
    	ylabel = {Время, с},
    	minor tick num = 0,
        xmin = 0,
        xmax = 30,
        width = 400,
        height = 320
    ]
    \addplot coordinates {
	    (1, 47.574) (8, 5.29) (27, 2.036876) 
    };
    \legend {dim = 1536 $\times$ 1536}
    \end{axis}
    \end{tikzpicture}
\end{center}

\hspace {2cm}

\begin{center}
    \begin{tikzpicture}
    \begin{axis}[
        scaled ticks = false,
        	xtick = {1, 8, 27},
        yticklabel style={
            /pgf/number format/precision=3,
            /pgf/number format/fixed,
            /pgf/number format/fixed zerofill,
          },
        legend pos = north east,
    	xlabel = {Количество процессов},
    	ylabel = {Время, с},
    	minor tick num = 0,
        xmin = 0,
        xmax = 30,
        width = 400,
        height = 320
    ]
    \addplot coordinates {
	    (1,  226.178548) (8, 48.01668) (27, 17.488632) 
    };
    \legend {dim = 3072 $\times$ 3072}
    \end{axis}
    \end{tikzpicture}
\end{center}

\end{document}
