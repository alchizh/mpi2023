\documentclass[12pt,a4paper,oneside,final]{article}
\usepackage{pgfplots}
% \pgfplotsset{width=7cm, compat=1.18}
\pgfplotsset{compat=1.9}
\usepackage[utf8]{inputenc}
\usepackage[russian]{babel}
\usepackage{graphicx} % \includegraphics
\usepackage{indentfirst}

\oddsidemargin = 0cm
\topmargin = -1.5cm
\textwidth = 16cm
\textheight = 24cm
\parindent = 0.5cm

\newcommand\Section[1]{
  \refstepcounter{section}
  \section*{\raggedright
    \arabic{section}. #1}
  \addcontentsline{toc}{section}{%
    \arabic{section}. #1}
}

\newcommand\Subsection[1]{
  \refstepcounter{subsection}
  \subsection*{\raggedright
    \arabic{section}.\arabic{subsection}. #1
  }
  \addcontentsline{toc}{subsection}{%
    \arabic{section}.\arabic{subsection}. #1}
}

\newcommand\Subsubsection[1]{
  \refstepcounter{subsubsection}
  \subsubsection*{\raggedright
    \arabic{section}.\arabic{subsection}.\arabic{subsubsection}. #1
  }
  \addcontentsline{toc}{subsubsection}{%
    \arabic{section}.\arabic{subsection}.\arabic{subsubsection}. #1}
}

\sloppy

\title{Задание 4 \\
  Параллельный алгоритм умножения матрицы на вектор\\
  Отчёт}
\author{Чижевский\,А.А.}
\date{2023}

\begin{document}

\maketitle

\Section{Постановка задачи}


Разработать параллельную программу с использованием технологии MPI, реализующую алгоритм умножения плотной матрицы на вектор $A \cdot b = c.$ $A \in Z^{m \times n}.$ Провести исследование эффективности разработанной программы на системе Polus.

\Section{Формат коммандной строки}

\begin{verbatim}
   mpisubmit.pl -p <P> ./prog -- <P_row> <P_col> <матрица_а> <вектор_b> <вектор_c>
\end{verbatim}

\Section{Спецификация системы}

\noindent

\noindent
Процессор: POWER8NVL 4023MHz 

\noindent
Размеры кэшей 

\noindent
L1d cache:             64K

\noindent
L1i cache:             32K

\noindent
L2 cache:              512K

\noindent
Размер строки кэша L1: 128 байт

\noindent
Число вычислительных ядер: 8

\Section{Результаты выполнения}

Размер матрицы А: $10000\times12000$

Размер вектора b: $12000\times1$

Ниже приведен график зависимости времени работы основного цикла программы (в секундах) от количества MPI процессов.

\begin{center}
    \begin{tikzpicture}
    \begin{axis}[
        scaled ticks = false,
        yticklabel style={
            /pgf/number format/precision=3,
            /pgf/number format/fixed,
            /pgf/number format/fixed zerofill,
          },
        legend pos = north west,
    	xlabel = {Режим},
    	ylabel = {Время, с},
    	minor tick num = 0,
        xmin = 0,
        xmax = 16,
        width = 400,
        height = 320
    ]
    \addplot coordinates {
	    (1, 1.016766) (2, 0.565827) (4, 0.299348) (8, 0.197953) (16, 0.169555) 
    };
    
    \end{axis}
    \end{tikzpicture}
\end{center}


\end{document}
